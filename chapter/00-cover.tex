%# -*- coding: utf-8-unix -*-

% 中文文档信息
\nuaaset{
  title = {基于深度学习的网络流量异常检测算法}, % 论文题目
  author = {闫珺},   % 作者
  studentid = {161420219},  % 学号(本科)
  college = {计算机科学与技术学院},    % 学院,或者(金城)院部
  major = {信息安全},     % 专业(本科)
  classid = {1614204},  % 班号(本科)
  advisors = {皮德常},  % 指导教师
  libraryclassid = {TP371},       % 中图分类号(硕博)
  subjectclassid = {080605},      % 学科分类号(硕博)
  thesisid = {1028704 17-S036},   % 论文编号(硕博)
  majorsubject = {学位论文排版},  % 学科、专业(硕博)
  researchfield = {排版引用},     % 研究方向(硕博)
  % applydate = {二〇一八年五月}  % 默认当前日期
}

% 英文文档信息(本科可以不用写)
\nuaasetEn{
  % college = {NUAA},
  title = {Deep Learning Based Network Traffic Anomaly Detection Algorithm},
  % majorsubject = {Thesis Typesetting},
  % author = {\nuaathesis~Group},
  % advisors = {Prof.~Donald Knuth},
  % applydate = {July, 2017}
}

%% 中文摘要
\begin{abstract}

随着网络的快速发展,网络犯罪所带来的损害越来越大,网络流量异常检测算法主要应用于网络入侵检测系统,是维护网络安全的一种有效的方法。然而网络流量的复杂性与高维度特点,使得开发一种有效的网络流量异常检测算法存在着许多挑战。深度学习的提出克服了这种困难,使得这一问题有了新的解决思路。因此,本文提出了一种基于深度学习的网络流量异常检测算法,首先介绍了深度学习和网络流量异常检测的基础理论,紧接着详细阐述了基于深度学习——DNN模型的关键性技术与思想。接下来,使用KDD Cup 99数据集的训练集和测试集对算法进行了仿真实验与测试,并进行了多组的对比实验选出最适合的模型参数,并对实验结果进行总结分析。结果表明,基于深度学习的网络流量异常检测可以克服网络流量数据的复杂性特点和难以训练的问题,检测准确率相比于浅层机器学习算法有着较大的的提升。

\end{abstract}
\keywords{深度学习, 机器学习, 网络流量异常检测, 网络入侵检测系统}

%% 英文摘要
\begin{abstractEn}
A brief example of paper.
\end{abstractEn}
\keywordsEn{Deep Learning, Machine Learning, Network Traffic Anomaly Detection, NIDS}
