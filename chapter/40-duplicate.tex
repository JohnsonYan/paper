\chapter{总结与展望}

\section{总结}

随着科技的进步,计算机硬件水平的不断提高,人工智能、机器学习、深度学习等技术在社会的方方面面发挥出自己强大的推动力。互联网在飞速发展的同时,网络安全问题也日益严重,对人们生产生活的影响也不断地变大。将深度学习应用于网络安全领域,是一项很有意义的任务。基于这种情况,本文提出了一种基于深度学习的网络流量异常检测算法。

通过认真研究网络流量异常检测相关内容,深入研究深度学习理论和深度神经网络的结构、特点、训练方法,我发现对于网络流量异常检测这种数据维度高、数据之间的关系复杂的问题,深度神经网络应该会有很不错的效果。因此,我将深度学习领域的深度神经网络应用于网络流量异常检测问题上,设计了一种基于深度学习的检测模型,并通过实验结合网络流量异常检测领域的知名数据集进行了实现与测试。实验选择了使用KDD 10\%的数据用于训练,使用包含未知类型攻击的corrected数据进行测试。在实验中,本文对深度神经网络的关键参数(网络结构、batchsize、优化算法等)进行了大量的对比实验,以期构建一个具有较高检测能力的深度学习模型。

实验结果也表明了深度学习在网络流量异常检测领域的价值,实验结果得到的92\%的准确率属于较为优秀的结果。同时鉴于一些研究中使用75\%训练数据用于训练,25\%的训练数据进行测试,本文在实验的最后一部分也给出了基本该模型的测试结果:99.92\%,取得了十分突出的准确率结果。

通过本文研究,深度学习与网络流量异常检测结合,有着非常不错的效果,可以考虑在真实情况的应用中,将网络的待检测流量整理后对本模型进行训练,预计训练好的模型在检测能力上会有着不错的效果。

\section{展望}

在本文的研究过程中,虽然在实验中取得了不错的效果,但仍然存在一些需要改进的地方。首先,深度学习目前仍是偏向于应用的领域,对模型的深度、参数没有一个明确的理论用于确定,本文所得到的参数都是基于经验以及大量的实验总结出来的。另外,实验选择了使用KDD Cup 99数据集模拟真实网络中的网络流量,但是该数据存在着数据分布不均匀的缺点,可能会对实验结果产生一定的影响,同时,该数据集于1999年被创造,虽然计算机网络流量的结构并未发生很大的变化,数据集中特征的类别与如今的网络环境差别不大,但是攻击类型、攻击方式已经发生了巨大的变化,当今流行的一些攻击手段在数据集中并不存在。因此,在未来的工作方向中,应该尝试构建基于真实网络的深度学习检测模型,使用符合目前网络安全环境的流量数据对模型训练是下一步的目标。深度学习领域的优秀算法有很多,本文只探讨了其中的一部分,使用其它深度学习的方法也是进一步的研究方向。