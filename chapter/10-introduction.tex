%# -*- coding: utf-8-unix -*-
\chapter{引言}\label{chap:intro}

这是南京航空航天大学(非官方)学位论文\LaTeX 模板,当前模板的版本是\version。本模板由\nuaathesis~Group共同开发,模板文档由Old Jack和yzwduck撰写。

本模板最早可以追溯到人人网上的一篇博客\footnote{\url{http://blog.renren.com/share/546499630/17959595570}},由黄大宁、邓欣珂、徐添豪、石坤四人共同开发完善,参考了当时东南大学的\seuthesix 模板;除此之外在Github上也可以找到一个repo\footnote{\url{https://github.com/naa803/nuaa-thesis}},由Felix Ding、Jun Wang、Jackie Hou三位老师和Vevi Zhong同学共同维护,但是repo中的.cls和.sty文件是空文件。该repo目前基于已停止更新的旧repo\footnote{\url{https://github.com/JackWzh/nuaathesis}},本模板的参考文献格式相比之更符合南航目前的参考文献标准,其他细节也更出色,故仍推荐使用我们的模板。

回顾人人网的模板,没有直接提供\verb|nuaa.png|和\verb|nuaa.bst|文件,可以使用强制编译的方法生成文件,但是缺少左上角南航字样,参考文献格式也不符合标准。除此之外,旧模板使用了已经被放弃使用的CJK宏包,因此在编译\verb+\Unicode{}+命令时会出错,代码的阅读性和维护性也不如现在的ctex和xeCJK。由于上述原因,许多初次使用\LaTeX 和使用经验不多的同学,在一开始就放弃了使用旧版模板进行毕业设计的书写及排版。

基于南航无可用\LaTeX 学位模板可用的现状,\nuaathesis~Group基于旧\oldnuaathesis 模板、现东南大学的\seuthesix 模板、上海交通大学的SJTU Thesis模板和重庆大学 CQUThesis 模板,进行了二次开发,基本实现了本硕博学位论文的模板。

现在\nuaathesis 模板的代码托管在Github\footnote{\url{https://github.com/nuaatug/nuaathesis}}上,如有修改建议或者其他要求欢迎在Github上开issue或提pull request,\nuaathesis~Group会尽快回复,并酌情处理您的需求。

本模板当前版本基于 Windows 平台 \TeX Live 2017 开发,并用 CI 在其他平台下测试,但尚未在Windows平台使用Mik\TeX 进行测试。
% 本模板基于Windows~10平台开发,使用MiKTeX v2.9发行版,所使用的宏包均跟进到最新版本。Linux平台由张一白使用\TeX Live测试,macOS平台由王成欣进行了测试,目前尚未出现任何问题。本模板尚未在Windows平台使用\CTeX / \TeX Live进行测试,
如出现问题,请自行Google、Bing、Baidu搜索解决方法。学会使用搜索引擎、熟练阅读外文是一个学生最基本的能力,更是一个\LaTeX 使用者得以立足和前进的根本。

\nuaathesis~Group非常欢迎其他南航的\LaTeX 使用者加入到本模板的开发与维护当中来,不断完善模板,为南航广大学子造福!

\section{研究背景与意义}
\subsection{研究背景}

异常检测指的是在数据中发现不符合正常行为或预期行为的模式的问题。这些不符合预期行为的模式在不同的应用领域通常被称为异常。现今,异常检测的应用非常广泛,例如网络入侵检测系统(NIDS)、信用卡欺诈检测、保险或医疗健康、安全等关键系统的故障检测以及对敌方军事活动的监视等领域。

赛门铁克旗下诺顿公司近日发布的《2017诺顿网络安全调查报告》显示,仅在去年一年,中国网络犯罪受害者的总损失就超过660亿美元,平均每人损失132美元,而每位受害者平均花费近4个工作日(28.3个小时)对攻击所带来的影响进行善后处理。随着互联网在中国的迅速普及,网络安全日益成为阻碍中国互联网发展的一股强大的阻力。网络入侵检测系统(NIDS)是一种有效的维护网络安全的技术,而NIDS的关键性技术就是网络流量异常检测,因此,如何高效准确的检测网络流量异常,是一项非常有意义且有挑战性的工作。

网络入侵检测系统(NIDS)是网络系统管理员检测组织内部网络各种安全漏洞的重要工具。NIDS监视,分析并发起进入或离开组织网络设备的网络流量的警报,以保证网络系统资源的机密性、完整性和可用性。这种NIDS不同于与被动阻拦外部攻击的防火墙技术,它是一种积极主动的安全防护技术,具有实时性、动态性和主动性等特点,能有效弥补其他静态防御工具的不足。基于入侵检测的方法,NIDS被分为两类:i)基于签名(误 用)的NIDS(SNIDS),以及ii)基于异常检测的NIDS(ADNIDS)。在SNIDS中,例如Snort,其原理是在NIDS中预先安装了定义好的攻击规则,针对安装的规则对流量进行模式匹配以检测网络中的入侵行为,也称为基于先验知识的入侵检测。SNIDS包括基于统计的方法、基于知识的方法和基于机器学习的方法。相比之下,ADNIDS只要观察到偏离了正常流量模式的流量,就会将网络流量归类为入侵,所以也被称之为基于行为的入侵检测系统。SNIDS可以有效检测已知的攻击,其优点是高检测准确度和低误报率。但是,由于规则必须预先安装在IDS中的限制,其表现在检测未知或新类型的攻击时会受到影响。另一方面,ADNIDS非常适合检测未知的、新类型的攻击。虽然ADNIDS容易产生较高的假阳性率,但它在识别新型攻击方面的理论潜力已经引起了研究界的广泛接受。

在为应对未知攻击开发有效和灵活的网络流量异常检测算法方面,主要存在两个挑战。首先,从网络流量数据集中选择用于异常检测的适当特征是困难的。随着攻击场景的不断变化和演变,针对一种类型的攻击所做的检测功能可能并不适用于其他类型的攻击。其次,在真实网络环境下的标签流量数据集很难用于开发NIDS。往往需要付出巨大的努力才能根据一段时间或者实时收集的原始网络流量痕迹生成这样的标签数据集,第二个挑战也正是因为这个原因而形成的。此外,为了保护组织内部网络结构的机密性以及用户的各种隐私信息,网络管理员通常不愿意报告其所管理的网络中可能发生的任何入侵情况。

目前,已经有多种机器学习技术被用于开发ADNIDS,例如人工神经网络(ANN),支持向量机(SVM),朴素贝叶斯 (NB),随机森林(RF),自组织映射(SOM)等等。异常网络流量检测在实现中主要被开发为分类器来区分普通流量和异常流量。许多NIDS会执行特征选择任务,以此从流量数据集中提取相关特征的子集以增强分类结果。特征选择可以去除冗余特征和噪声,有助于降低因冗余特征和噪声导致错误训练的可能性。最近,基于深度学习的方法已经被成功应用于音频、图像和语音处理应用中。这些深度学习方法旨在通过深层网络的复杂抽象来获取到一个良好的数据特征表示,之后,将这些经过复杂抽象的特征应用于监督学习的分类任务中,以此克服浅层机器学习本身存在的问题。可以想象,基于深度学习的方法可以帮助克服开发有效的网络流量异常检测算法的挑战,因此,我选择了深度学习算法来尝试解决网络流量异常检测这一问题。我通过将深度学习模型应用于KDD Cup 99入侵检测数据集来验证基于深度学习的网络流量异常检测算法的可行性。通过多组对比实验寻找用于网络流量异常检测的深度学习模型的合适参数,同时也提供了该深度学习模型与其他技术的比较。

\subsection{研究意义}

我们推荐使用根目录下的\verb|.bat|、\verb|.sh|脚本进行编译。若需要手动编译,\textbf{切记使用\XeLaTeX 进行编译。}使用\XeLaTeX 和\hologo{BibTeX}在文档中加入参考文献的流程可参考如下命令:

\iffalse
\begin{lstlisting}[basicstyle=\small\ttfamily, caption=手动逐次编译, numbers=none]
xelatex -no-pdf .tex文件名
bibtex .tex文件名
xelatex .tex文件名
xelatex .tex文件名
\end{lstlisting}
\else
\begin{verbatim}
xelatex -no-pdf .tex文件名
bibtex .tex文件名
xelatex .tex文件名
xelatex .tex文件名
\end{verbatim}

注:因为已知bug,请在验证自己的环境后再使用 \texttt{lstlisting} 包,参见\url{https://github.com/CTeX-org/ctex-kit/issues/297}。

\fi

使用\XeLaTeX 引擎编译可以直接通过各\LaTeX 编辑器实现,如:TeXworks,TeXmaker,TeXStudio,Emacs+插件,Atom+插件等等。biber命令需使用Windows的cmd/Power Shell、Linux和macOS下的bash实现。

提示:Windows命令行用户,可以使用\verb|chcp 65001| 命令,将字符编码切换成 UTF-8,避免输出乱码。Windows 7系统及更高版本可使用 PowerShell 避免输出乱码。

使用biber需注意:\textbf{\texttt{.bib} 文件内的文件记录必须在 \texttt{.tex} 文件中被引用,不引用的记录不要因为懒而不去除,否则将编译失败。}

目录内容需要编译两次才能正常显示,原因推断为早期的电脑内存不够,所以将目录的生成分成了两步来进行。

\subsection{模板文件结构}
\subsubsection{必需文件}
\begin{itemize}[noitemsep,topsep=0pt,parsep=0pt,partopsep=0pt]
  \item \verb|.tex|文件:\TeX 主文件,chapter文件夹下有各个章节的文件,强烈建议将文章模块化,方便调试与版本管理;
  \item \verb|.bst|、\verb|.dtx|、\verb|.ins|文件:模板定义文件,不可删除;
  \item \verb|.bib|文件:参考文献数据库文件;
  \item \verb|.bat|、\verb|.sh|文件:批处理脚本,Windows系统下双击\verb|\bat|使用,Linux \& macOS 可在终端中使用\verb|.sh|;
  \item figure文件夹:存放要插入的图片,请不要删改已有的文件,这些文件是模板的一部分
\end{itemize}

\subsubsection{高级用户文件}
\begin{itemize}[noitemsep,topsep=0pt,parsep=0pt,partopsep=0pt]
  \item \verb|.ci|、\verb|.circleci|文件夹:持续集成配置文件夹,有需求的同学可直接使用,无需求的同学可以直接删除;
  \item \verb|.appveyor.yml|、\verb|.travis.yml|文件:持续集成配置文件,有需求的同学可直接使用,无需求的同学可以直接删除;
  \item \verb|.gitignore|文件:Git版本管理系统配置文件,有需求的同学可直接使用,无需求的同学可以直接删除
\end{itemize}