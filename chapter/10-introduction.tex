%# -*- coding: utf-8-unix -*-
\chapter{引言}\label{chap:intro}

\section{研究背景和研究意义}

赛门铁克旗下诺顿公司近日发布的《2017诺顿网络安全调查报告》显示,仅在去年一年,中国网络犯罪受害者的总损失就超过660亿美元,平均每人损失132美元,而每位受害者平均花费近4个工作日(28.3个小时)对攻击所带来的影响进行善后处理。据有关媒体统计,2017年上半年泄露或被盗的数据已经达到了惊人的19亿条。值得一提的是,这一数据已经是2016年泄露数据的总量,全年预计将超过50亿条,其中,仅仅雅虎一家就泄露了多达30亿条。同时,在2017上半年中,全球总共发生了918起网络安全入侵事件,我国用户的信息每年至少被泄露五次。可见,随着互联网的迅速普及,网络安全事件也呈现倍速上升的趋势,它涉及的范围非常广泛,重点主要集中在高校、医疗行业、政府机构以及金融行业等。在互联网在我国急速普及的趋势下,越来越多的人加入了互联网的浩瀚世界,同时也意味承担了这复杂网络世界的风险,网络安全日益成为阻碍中国互联网发展的一股强大的阻力。
网络入侵检测系统(NIDS)是一种有效的维护网络安全的技术,而NIDS的关键性技术就是网络流量异常检测,异常检测指的是在数据中发现不符合正常行为或预期行为的模式的问题。这些不符合预期行为的模式在不同的应用领域通常被称为异常。现今,异常检测的应用非常广泛,例如网络入侵检测系统(NIDS)、信用卡欺诈检测、保险或医疗健康、安全等关键系统的故障检测以及对敌方军事活动的监视等领域。

网络入侵检测系统(NIDS)是网络系统管理员检测组织内部网络各种安全漏洞的重要工具。NIDS监视,分析并发起进入或离开组织网络设备的网络流量的警报,以保证网络系统资源的机密性、完整性和可用性。这种NIDS不同于与被动阻拦外部攻击的防火墙技术,它是一种积极主动的安全防护技术,具有实时性、动态性和主动性等特点,能有效弥补其他静态防御工具的不足。基于入侵检测的方法,NIDS被分为两类:i)基于签名(误 用)的NIDS(SNIDS),以及ii)基于异常检测的NIDS(ADNIDS)。在SNIDS中,例如Snort,其原理是在NIDS中预先安装了定义好的攻击规则,针对安装的规则对流量进行模式匹配以检测网络中的入侵行为,也称为基于先验知识的入侵检测。SNIDS包括基于统计的方法、基于知识的方法和基于机器学习的方法。相比之下,ADNIDS只要观察到偏离了正常流量模式的流量,就会将网络流量归类为入侵,所以也被称之为基于行为的入侵检测系统。SNIDS可以有效检测已知的攻击,其优点是高检测准确度和低误报率。但是,由于规则必须预先安装在IDS中的限制,其表现在检测未知或新类型的攻击时会受到影响。另一方面,ADNIDS非常适合检测未知的、新类型的攻击。虽然ADNIDS容易产生较高的假阳性率,但它在识别新型攻击方面的理论潜力已经引起了研究界的广泛接受。

在为应对未知攻击开发有效和灵活的网络流量异常检测算法方面,主要存在两个挑战。首先,从网络流量数据集中选择用于异常检测的适当特征是困难的。随着攻击场景的不断变化和演变,针对一种类型的攻击所做的检测功能可能并不适用于其他类型的攻击。其次,在真实网络环境下的标签流量数据集很难用于开发NIDS。往往需要付出巨大的努力才能根据一段时间或者实时收集的原始网络流量痕迹生成这样的标签数据集,第二个挑战也正是因为这个原因而形成的。此外,为了保护组织内部网络结构的机密性以及用户的各种隐私信息,网络管理员通常不愿意报告其所管理的网络中可能发生的任何入侵情况。

目前,已经有多种机器学习技术被用于开发ADNIDS,例如人工神经网络(ANN),支持向量机(SVM),朴素贝叶斯 (NB),随机森林(RF),自组织映射(SOM)等等。异常网络流量检测在实现中主要被开发为分类器来区分普通流量和异常流量。许多NIDS会执行特征选择任务,以此从流量数据集中提取相关特征的子集以增强分类结果。特征选择可以去除冗余特征和噪声,有助于降低因冗余特征和噪声导致错误训练的可能性。最近,基于深度学习的方法已经被成功应用于音频、图像和语音处理应用中。这些深度学习方法旨在通过深层网络的复杂抽象来获取到一个良好的数据特征表示,之后,将这些经过复杂抽象的特征应用于监督学习的分类任务中,以此克服浅层机器学习本身存在的问题。可以想象,基于深度学习的方法可以帮助克服开发有效的网络流量异常检测算法的挑战,因此,我选择了深度学习算法来尝试解决网络流量异常检测这一问题。我通过将深度学习模型应用于KDD Cup 99入侵检测数据集来验证基于深度学习的网络流量异常检测算法的可行性。通过多组对比实验寻找用于网络流量异常检测的深度学习模型的合适参数,同时也提供了该深度学习模型与其他技术的比较。

\section{国内外研究现状}
\section{论文的研究内容}
\section{论文的组织结构}
\section{本章小结}